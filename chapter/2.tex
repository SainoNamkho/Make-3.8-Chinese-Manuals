\chapter{GNU make 介绍}

make在执行时,需要一个命名为Makefile的文件。这个文件告诉make以何种方式编译源代码和链接程序。典型地,可执行文件由一些.o文件按照一定的顺序生成或者更新。如果在工程中已经存在一个或者多个正确的Makefile,当修改了工程中的若干源文件以后,而需要根据修改来更新可执行文件或者库文件,正如前面提到的只需要在shell下执行“make”。根据源文件的修改情况,所对应.o文件将被更新、使用(包含)这个.o的库文件也会被更新、最终可执行程序也将被更新。

make通过比较对应(相关)文件(规则的目标和依赖)的最后修改时间,来决定哪些文件需要更新、那些文件不需要更新。对需要更新的文件make就执行数据库中所记录的相应命令(在make读取Makefile以后会建立一个编译过程的描述数据库。此数据库中记录了各个文件之间的相互关系,以及它们的关系描述)来重建它,对于不需要重建的文件make什么也不做。

并且,如果需要单独更新一个文件,则可以通过make的命令行选项来指定具体文件,而不需要对整个工作执行make操作。可参考 \hyperref[sec:9.2]{9.2 指定终极目标} 一节


\section{Makefile简介}
在执行make之前,需要一个命名为Makefile的特殊文件(本文的后续将使用Makefile作为这个特殊文件的文件名)来告诉make你的目的(完成什么任务)、以及如何实现这个目的(完成任务的步骤及方法)。通常,make工具主要被用来进行工程编译并产生最终的目标文件(由工程的源文件产生必要的库文件,最终生成可执行文件)。Makefile文件的内容描述了工程中所有源文件的结构关系、以及由这些源文件生成最终可执行文件的方法。这里可执行文件不仅包括那些可以在特定操作系统上被加载执行的文件,这类文件通常具有特定的格式(elf格式);也包括那些能够被特定CPU执行的指令序列集合文件,这些文件无特定格式,可看做一个指令序列的集合(嵌入式中烧写到Flash中用于系统启动的bootloader image文件)。

本节将分析一个简单的Makefile,它描述了一个工程中的8个C源文件和三个头文件的相互关系,以及如何由这些源文件产生最终的可执行文件。这个Makefile提供了创建最终可执行程序(例子中的edit文件)所必需的全部信息,make程序根据Makefile中的描述(规则)执行相关的命令来完成给定的任务(如:编译、链接和清除编译过程文件等)。后续会慢慢接触到复杂的Makefile,但一切还是从要从简单入手。

当使用make工具进行工程编译时,根据源文件或者头文件的修改情况,make决定是否对源文件进行编译(或者重新编译),并决定是否重新链接产生最终的可执行文件。规则如下:
\begin{enumerate}
\itemsep=0pt \parskip=0pt
  \item \textbf{所有的源文件没有被编译过,则对各个C源文件进行编译并进行链接,生成最后的可执行程序;}
  \item \textbf{每一个在上次执行make执行结束后被修改过的C源代码文件在本次执行make时将会被重新编译;}
  \item \textbf{头文件在上一次make执行结束后被修改。则所有包含此头文件的C源文件在本次执行make时将会被重新编译。}
\end{enumerate}

后两种情况是make只将修改过的C源文件重新编译生成.o文件,对于没有修改的文件不进行任何工作。重新编译过程中,任何一个源文件的修改将产生新的对应的.o文件,新的.o文件将和以前的已经存在、此次没有重新编译的.o文件重新连接生成最后的可执行程序。

首先让我们先来看一些Makefile相关的基本知识。


\section{Makefile规则介绍}
一个简单的Makefile描述规则格式:
\begin{Verbatim}[]
TARGET... : PREREQUISITES...
    COMMAND
    ...
    ...
\end{Verbatim}

\textbf{target}:规则的目标。通常是最后需要生成的文件名或者为了实现这个目的而必需的中间过程文件名。可以是.o文件、也可以是最后的可执行程序的文件名等。另外,目标也可以是一个make执行动作的名称,如目标“clean”,我们称这样的目标是“伪目标”。参考 \hyperref[sec:4.6]{4.6 Makefile伪目标} 一节

\textbf{prerequisites}:规则的依赖。生成规则目标所需要的文件名列表。通常一个目标依赖于一个或者多个文件。

\textbf{command}:规则的命令行。是规则所要执行的动作(任意的shell命令或者是可在shell下执行的程序)。它限定了make执行这条规则时所执行的动作。

一个规则可以有多个命令行,每一条命令占一行。

\begin{quote}\kaishu
  \textbf{注意}:每一个命令行必须以[Tab]字符开始,[Tab]字符告诉make此行是一个命令行,make按照命令完成相应的动作。
\end{quote}


在Makefile中“规则”就是描述在什么情况下、如何重建规则的目标(可能是一个文件),通常规则包含了目标的依赖关系(目标的依赖文件)和重建目标的命令行。make执行重建目标的命令,来创建或者重建规则的目标(此目标也可以是触发这个规则被执行的规则的依赖)。\textbf{规则包含了文件之间的依赖关系和更新此规则目标所需要的命令。}
一个Makefile文件中通常还包含了除规则以外的很多东西(后续我们会一步一步的展开)。一个最简单的Makefile可能只包含规则。规则在有些Makefile中可能看起来非常复杂,但是无论规则的书写是多么的复杂,它都符合规则的基本格式。
make程序根据规则的依赖关系,决定是否执行规则所定义的命令的过程我们称之为规则的执行或者简称\textbf{执行规则。}


\section{简单的示例}
本小节开始我们在第一小节中提到的例子。此例子由3个头文件和8个C文件组成。我们将书写一个简单的Makefile,来描述如何创建最终的可执行文件“edit”,此可执行文件依赖于8个C源文件和3个头文件。Makefile文件的内容如下:

\begin{Verbatim}[]
#sample Makefile
edit : main.o kbd.o command.o display.o \
    insert.o search.o files.o utils.o
    cc -o edit main.o kbd.o command.o display.o \
        insert.o search.o files.o utils.o
main.o : main.c defs.h
    cc -c main.c
kbd.o : kbd.c defs.h command.h
    cc -c kbd.c
command.o : command.c defs.h command.h
    cc -c command.c
display.o : display.c defs.h buffer.h
    cc -c display.c
insert.o : insert.c defs.h buffer.h
    cc -c insert.c
search.o : search.c defs.h buffer.h
    cc -c search.c
files.o : files.c defs.h buffer.h command.h
    cc -c files.c
utils.o : utils.c defs.h
    cc -c utils.c
clean :
    rm edit main.o kbd.o command.o display.o \
        insert.o search.o files.o utils.o
\end{Verbatim}

首先书写时,可以将一个较长行使用反斜线(\verb"\")来分解为多行,这样可以使我们的Makefile书写清晰、容易阅读理解。但需要注意:反斜线之后不能有空格(这也是大家最容易犯的错误,错误比较隐蔽)。我们推荐将一个长行分解为使用反斜线连接得多个行的方式。在完成了这个Maekfile以后;需要创建可执行程序“edit”,所要做的就是在包含此Makefile的目录(当然也在代码所在的目录)下输入命令“make”。删除已经此目录下之前使用“make”生成的文件(包括那些中间过程的.o文件),也只需要输入命令“make clean”就可以了。

在这个Makefile中,我们的目标(target)就是可执行文件“edit”和那些.o文件(main.o,kbd.o….);依赖(prerequisites)就是冒号后面的那些 .c 文件和 .h文件。所有的.o文件既是依赖(相对于可执行程序edit)又是目标(相对于.c和.h文件)。命令包括 “\verb"cc –c maic.c"”、“\verb"cc –c kbd.c"”……

当规则的目标是一个文件,在它的任何一个依赖文件被修改以后,在执行“make”时这个目标文件将会被重新编译或者重新连接。当然,此目标的任何一个依赖文件如果有必要则首先会被重新编译。在这个例子中,“edit”的依赖为8个.o文件;而“main.o”的依赖文件为“main.c”和“defs.h”。当“main.c”或者“defs.h”被修改以后,再次执行“make”,“main.o”就会被更新(其它的.o文件不会被更新),同时“main.o” 的更新将会导致“edit”被更新。

在描述依赖关系行之下通常就是规则的命令行(存在一些些规则没有命令行),命令行定义了规则的动作(如何根据依赖文件来更新目标文件)。\textbf{命令行必需以[Tab]键开始,以和Makefile其他行区别。就是说所有的命令行必需以[Tab] 字符开始,但并不是所有的以[Tab]键出现行都是命令行。但make程序会把出现在第一条规则之后的所有以[Tab]字符开始的行都作为命令行来处理。}(记住:make程序本身并不关心命令是如何工作的,对目标文件的更新需要你在规则描述中提供正确的命令。“make”程序所做的就是当目标程序需要更新时执行规则所定义的命令)。

目标“clean”不是一个文件,它仅仅代表执行一个动作的标识。正常情况下,不需要执行这个规则所定义的动作,因此目标“clean”没有出现在其它任何规则的依赖列表中。因此在执行make时,它所指定的动作不会被执行。除非在执行make时明确地指定它。而且目标“clean”没有任何依赖文件,它只有一个目的,就是通过这个目标名来执行它所定义的命令。\textbf{Makefile中把那些没有任何依赖只有执行动作的目标称为“伪目标”(phony targets)}。需要执行“clean”目标所定义的命令,可在shell下输入:make clean。


\section{make如何工作}

默认的情况下,make执行的是Makefile中的第一个规则,此规则的第一个目标称之为“最终目的”或者“终极目标”(就是一个Makefile最终需要更新或者创建的目标, 参考 \hyperref[sec:9.2]{9.2 指定终极目标} 一节)。

上例的Makefile,目标“edit”在Makefile中是第一个目标,因此它就是make的“终极目标”。当修改了任何C源文件或者头文件后,执行make将会重建终极目标“edit”。

当在shell提示符下输入“make”命令以后。make读取当前目录下的Makefile文件,并将Makefile文件中的第一个目标作为其执行的“终极目标”,开始处理第一个规则(终极目标所在的规则)。在我们的例子中,第一个规则就是目标“edit”所在的规则。规则描述了“edit”的依赖关系,并定义了链接.o文件生成目标“edit”的命令; make在执行这个规则所定义的命令之前,首先会处理目标“edit”的所有的依赖文件(例子中的那些.o文件)的更新规则(以这些.o文件为目标的规则)。对这些.o文件为目标的规则处理有下列三种情况:
\begin{enumerate}
\itemsep=0pt \parskip=0pt
    \item 目标.o文件不存在,使用其描述规则创建它;
    \item 目标.o文件存在,目标.o文件所依赖的.c源文件、.h文件中的任何一个比目标.o文件“更新”(在上一次make之后被修改)。则根据规则重新编译生成它;
    \item 目标.o文件存在,目标.o文件比它的任何一个依赖文件(.c源文件、.h文件)“更新”(它的依赖文件在上一次make之后没有被修改),则什么也不做。
\end{enumerate}

这些.o文件所在的规则之所以会被执行,是因为这些.o文件出现在“终极目标”的依赖列表中。在Makefile中一个规则的目标如果不是“终极目标”所依赖的(或者“终极目标”的依赖文件所依赖的),那么这个规则将不会被执行,除非明确指定执行这个规则(执行make时,在命令行指定具体目标,则这个目标所在的规则会被执行,例如 “make clean”)。在编译或者重新编译生成一个.o文件时,make同样会去寻找它的依赖文件的重建规则(这样的一个规则:这个依赖文件在规则中作为目标出现),在这里就是.c和.h文件的重建规则。在上例的Makefile中没有哪个规则的目标是.c或者.h文件,所以没有重建.c和.h文件的规则(这些文件本身已经存在,如果不存在则在make时会出现错误提示)。不过C言语源程序文件可以使用工具Bison或者Yacc来生成(具体用法可参考相应的手册)。

完成了对.o文件的创建(第一次编译)或者更新之后,make程序将处理终极目标“edit”所在的规则,分为以下三种情况:
\begin{enumerate}
    \itemsep=0pt \parskip=0pt
    \item 目标文件“edit”不存在,则执行规则以创建目标“edit”。
    \item 目标文件“edit”存在,其依赖文件中有一个或者多个文件比它“更新”,则根据规则重新链接生成“edit”。
    \item 目标文件“edit”存在,它比它的任何一个依赖文件都“更新”,则什么也不做。
\end{enumerate}

上例中,如果更改了源文件“insert.c”后执行make,“insert.o”将被更新,之后终极目标“edit”将会被重生成;如果我们修改了头文件“command.h”之后运行“make”,那么“kbd.o”、“command.o”和“files.o”将会被重新编译,之后同样终极目标“edit”也将被重新生成。

以上我们通过一个简单的例子,介绍了Makefile中目标和依赖的关系。简单总结如下:对一个Makefile文件,在执行make时,make程序先解析终极目标所在的规则(上节例子中的第一个规则),根据其依赖文件(例子中第一个规则的8个.o文件)依次(按照依赖文件列表从左到右的顺序)寻找创建这些文件的规则。首先为第一个依赖文件(main.o)寻找创建规则,如果第一个依赖文件依赖于其它文件(main.c、defs.h),则同样为这个依赖文件寻找创建规则(创建main.c和defs.h的规则,通常源文件和头文件已经存在,也不存在重建它们的规则)……,直到为所有的依赖文件找到合适的创建规则。之后make从最后一个规则(上例目标为main.o的规则)回退开始执行,最终完成终极目标的第一个依赖文件的创建和更新。之后对第二个、第三个、第四个……终极目标的依赖文件执行同样的过程(上例的的顺序是“main.o”、“kbd.o”、“command.o”……)。

创建或者更新任何一个规则依赖文件的过程都是这样的一个过程(类似于c语言中的递归过程)。执行规则时按照依赖文件列表的罗列顺序,依次对规则的每一个依赖文件,使用同样方式(按照同样的过程)去重建它,在完成对所有依赖文件的重建之后,最后一步才是重建此规则的目标。

更新(或者创建)终极目标的过程中,如果任何一个规则执行出现错误make就立即报错并退出。整个过程make只是负责执行规则,而对具体规则所描述的依赖关系的正确性、规则所定义的命令的正确性不做任何判断。就是说,一个规则的依赖关系是否正确、描述重建目标的规则命令行是否正确,make不做任何正确性检查。

因此,要正确的编译一个工程。需要书写Makefile的程序员去保证所有规则依赖关系的正确性、执行命令的正确性。执行make时make不会进行诸如此类问题正确性的验证和假设,它所做的仅仅是严格按照所提供规则的依赖关系,决定是否需要完成此规则所定义的动作(执行规则所定义的命令)。


\section{指定变量}
同样是上边的例子,我们来看一下终极目标“edit”所在的规则:

\begin{Verbatim}[]
edit : main.o kbd.o command.o display.o \
    insert.o search.o files.o utils.o
    cc -o edit main.o kbd.o command.o display.o \
        insert.o search.o files.o utils.o
\end{Verbatim}

在这个规则中.o文件列表出现了两次;第一次:作为目标“edit”的依赖文件列表出现,第二次:规则命令行中作为“cc”的参数列表。对于这个规则,如果需要为目标“edit”增加一个的依赖文件,就需要在两个地方添加(依赖文件列表和规则的命令中)。添加时可能在“edit”的依赖列表中加入了、但忘记了给命令行中添加,或反之。这给后期的维护和修改带来不便,添加或删除一个文件需修改多处,难免出现遗漏。

为了避免这个问题,在实际工作中大家都比较认同的方法是,使用一个变量“objects”、“OBJECTS”、“objs”、“OBJS”、“obj”或者“OBJ”来代表必须的.o文件列表。在Makefile中使用到这些文件的地方,用这个变量来代替。在上例的Makefile中可以在Makefile文件中增加这样一行(定义一个变量objects):

\begin{Verbatim}[]
objects = main.o kbd.o command.o display.o \
    insert.o search.o files.o utils.o
\end{Verbatim}

“objects”作为一个变量,它代表所需要的所有.o文件列表。在定义了此变量后,就可以在需要使用这些.o文件列表的地方用“\$(objects)”来表示它,不再需要罗列一长串.o文件(变量可参考 第六章 使用变量 )。上例的规则就可以这样写:

\begin{Verbatim}[]
objects = main.o kbd.o command.o display.o \
    insert.o search.o files.o utils.o
edit : $(objects)
    cc -o edit $(objects)
……
……
clean :
    rm edit $(objects)
\end{Verbatim}


这样,需要为终极目标“edit”增删.o依赖时,只需改变“objects”的定义(增删.o文件)。这样不但可以减少书写工作量,而且可以减小修改产生错误的几率。

\section{自动推导规则}
在使用make编译.c源文件时,编译.c源文件规则的命令可以不用明确给出。这是因为make本身存在一个默认的规则,能够自动完成对.c文件的编译并生成对应的.o文件。它执行命令“cc -c”来编译.c源文件。在Makefile中我们只需要给出需要重建的目标文件名(一个.o文件),make会自动为这个.o文件寻找合适的依赖文件(对应的.c文件。对应是指:文件名除后缀外,其余都相同的两个文件),并使用正确的命令来重建这个目标文件。对于上边的例子,此默认规则就使用命令“cc -c main.c -o main.o”来创建文件“main.o”。对一个目标文件是“N.o”,倚赖文件是“N.c”的规则,完全可以省略其规则的命令行,而由make使用默认命令。此默认规则称为make的隐含规则(关于隐含规则可参考 第十章 使用隐含规则)

这样,在书写Makefile时,就可以省略掉描述.c文件和.o依赖关系的所有规则,只需要给出那些特定的规则描述(.o目标所需要的.h文件,make不存在.o文件和.h文件对应的隐含规则)。因此上边的例子就有了更加简单的书写方式,同样使用变量“objects”。Makefile内容如下:


\begin{Verbatim}[]
# sample Makefile
objects = main.o kbd.o command.o display.o \
    insert.o search.o files.o utils.o

edit : $(objects)
    cc -o edit $(objects)

main.o : defs.h
kbd.o : defs.h command.h
command.o : defs.h command.h
display.o : defs.h buffer.h
insert.o : defs.h buffer.h
search.o : defs.h buffer.h
files.o : defs.h buffer.h command.h
utils.o : defs.h

.PHONY : clean
clean :
    rm edit $(objects)
\end{Verbatim}

这种格式的Makefile更为接近于实际应用。(关于目标“clean”的详细说明在后续章节。参考 \hyperref[sec:4.6]{4.6 Makefile伪目标} 一节 和 \hyperref[sec:5.4]{5.4 命令的错误} 一节)

make的隐含规则在实际工程的Makefile中经常使用,它使得编译过程变得方便。几乎所有的Makefile中都用到了make的隐含规则,make的隐含规则是非常重要的一个概念。第十章将对会对隐含规则进行详细讨论。


\section{另类风格的makefile}
上一节中提到过,在Makefile中,所有的.o目标文件都可以使用隐含规则由make来自动重建。根据这一点,我们书写了更加简洁的Makefile,书写式规则是按照目标进行分类(目标决定依赖)。而在下面的这个Makefile,是按照依赖(依赖决定目标)而不是目标对规则进行分组分类,形成另外一种风格的Makefile。如下:

\begin{Verbatim}[]
#sample Makefile
objects = main.o kbd.o command.o display.o \
    insert.o search.o files.o utils.o

edit : $(objects)
    cc -o edit $(objects)

$(objects) : defs.h
kbd.o command.o files.o : command.h
display.o insert.o search.o files.o : buffer.h
\end{Verbatim}

本例中,我们以三个头文件为出发点,对依赖于每一个头文件的目标进行合并。书写出一个多目标规则,规则中多个目标同时依赖于对应的一个头文件,而且同一个文件可能同时存在多个规则中。例子中头文件“defs.h”作为所有.o文件的依赖文件。其它两个头文件作为部分.o文件(所在规则中的多个目标文件)的依赖文件。

这种风格的Makefile并不值得我们借鉴。问题在于:同时把多个目标文件的依赖放在同一个规则中进行描述(一个规则中含有多个目标文件),这样导致规则定义不明了,比较混乱。建议书写Makefile时避免采用这种方式(除非有非常充分的理由)。一旦采用这种方式为工程书写Makefile,那么后期维护将会是一件非常痛苦的事情。

书写规则建议的方式是:\textbf{单目标,多依赖。}就是说尽量要做到一个规则中只存在一个目标文件,此规则可以有多个依赖文件。避免使用多目标,单依赖的方式。遵循这个规则的Makefile不但清晰、明了,而且便于后期维护。

\section{清除工作目录过程文件}

使用规则不但可以完成对源代码的编译,也可以实现其它任务。例如:前边提到实现清除编译工程时产生的文件、或者编译的过程文件(edit和哪些.o文件)的规则:
\begin{Verbatim}[]
clean :
    rm edit $(objects)
\end{Verbatim}
实际应用中,此规则可能被写成如下稍微复杂一些的样子
\begin{Verbatim}[]
.PHONY : clean
clean :
    -rm edit $(objects)
\end{Verbatim}

此两种实现有两点不同: 1. 通过“.PHONY”特殊目标将“clean”目标声明为伪目标。避免当磁盘上存在一个名为“clean”文件时,目标“clean”所在规则的命令无法执行(参考 \hyperref[sec:4.6]{4.6 Makefile伪目标} 一节)。2. 在命令行之前使用“-”,意思是忽略命令“rm”的执行错误(参考 \hyperref[sec:5.4]{5.4 命令的错误} 一节)。

这样的一个目标在Makefile中不能将其作为终极目标(Makefile的第一个目标)。因为这个规则的初衷并不是在你输入了make后执行删除动作,而是要创建或者更新一个文件(库或者可执行程序)。上例中,就是在输入make以后要需要对目标“edit”进行创建或者重建。

上例中因为目标“clean”没有出现在终极目标“edit”依赖列表中(终极目标的直接依赖或者间接依赖),所以在执行“make”时,目标“clean”所在的规则将不会被执行。需要执行此规则,要在make的命令行中明确指定这个目标(执行“make clean”)来执行它。关于make指定执行目标可参考 \hyperref[sec:9.2]{9.2 指定终极目标} 一节。
